\documentclass{article}
\usepackage[utf8]{inputenc}
\usepackage{tikz}
\usepackage{amsmath}

\title{Assignment4}
\author{Swapnil Sirsat}
\date{January 2021}

\begin{document}
\maketitle
\section*{Question: }
Draw $\Delta$ABC with a = 7, $\angle$B = 45$^\circ$
and $\angle$A = 105$^\circ$
\section*{Answer: }
To construct $\Delta$ABC we first need to find $\angle$C\\
By angle sum property we know that
\begin{align}
    \angle A + \angle B + \angle C = 180^\circ
\end{align}
putting values of $\angle$A and $\angle$B in equation 1 we get
\begin{gather*}
    45^\circ + 105^\circ + \angle C = 180^\circ\\
        \angle C = 180^\circ - 150^\circ \\
        \angle C = 30^\circ
\end{gather*}
Steps of Construction: 
\begin{enumerate}
    \item Draw a line segment BC of length = 7 units
    \item Using a protractor at point C, draw a line CX making an angle of 30$^\circ$ with CB
    \item Similarly, from point C draw a line BY making an angle of 45$^\circ$ with BC
    \item Mark the point of intersection of CX and BY as A
    \item figure ABC is the required triangle
\end{enumerate}
\newpage
\begin{tikzpicture}
\draw[gray, thick](0,0) -- (7,0);
\draw[gray, thick](0,0) -- (5,5);
\draw[gray, thick](7,0) -- (0,4.04);
\node at(0,0)[below left]{$B$};
\node at(3.5,0)[below left]{$a = 7units$};
\node at(7,0)[below right]{$C$};
\node at(0,4.04)[above left]{$X$};
\node at(5,5)[above right]{$Y$};
\filldraw[black] (2.56,2.56) circle (2pt) node[anchor=west] {A};

\end{tikzpicture}


\end{document}
